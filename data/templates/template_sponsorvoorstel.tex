%% GEWIS Letterhead Template
%% Stijl
%% All documentation can be found on https://gewis.nl/~stijl/

%%      _____                _             __                     
%%    |  _  |              | |           / _|_                   
%%   | | | |_   _____ _ __| | ___  __ _| |_(_)                  
%%  | | | \ \ / / _ \ '__| |/ _ \/ _` |  _|                    
%% \ \_/ /\ V /  __/ |  | |  __/ (_| | |  _                   
%% \___/  \_/ \___|_|  |_|\___|\__,_|_| (_)                  
%%                                                           
%%      _____ _ _      _                                          
%%    /  __ \ (_)    | |                                    _    
%%   | /  \/ |_  ___| | __  _ __ ___   ___ _ __  _   _   _| |_  
%%  | |   | | |/ __| |/ / | '_ ` _ \ / _ \ '_ \| | | | |_   _| 
%% | \__/\ | | (__|   <  | | | | | |  __/ | | | |_| |   |_|   
%% \____/_|_|\___|_|\_\ |_| |_| |_|\___|_| |_|\__,_|         
%%                                                           
%%       _____                                     _           _   
%%     /  __ \                                   (_)         | |  
%%    | /  \/ ___  _ __  _   _   _ __  _ __ ___  _  ___  ___| |_ 
%%   | |    / _ \| '_ \| | | | | '_ \| '__/ _ \| |/ _ \/ __| __|
%%  | \__/\ (_) | |_) | |_| | | |_) | | | (_) | |  __/ (__| |_ 
%%  \____/\___/| .__/ \__, | | .__/|_|  \___/| |\___|\___|\__|
%%            | |     __/ | | |            _/ |              
%%           |_|    |___/  |_|           |__/               
%% Using overleaf? Click on menu and then on copy project


\documentclass[
 digital,         % Use toggle digital to include letterhad in exported pdf
 %printletterhead,% Use toggle printbackground to print letterhead in exported pdf - implies digital
 %english         % Use toggle English for having the template printed in English. You might need to  all compiled files to prevent having errors after switching the language
]{GEWISLetter}

%% This example letter requires version 0.4 of the GEWISLetter package, which was released on November 27, 2019
\makeatletter
    \@ifclasslater{GEWISLetter}{2019/11/27}{}{
        \ClassError{GEWISLetter}{Dit voorbeeld vereist GEWISLetter v0.4 of hoger}%
    }
\makeatother

%% In this example we show some Lorem Ipsum; it is not neccessary to include this package if you are not going to use it
\usepackage{lipsum}

%% In case we want to use a different sender or contact details, we set it here
%\setSenderName{Ouderdagcommissie}
%\setSenderMail{odc@gewis.nl}
%\setSenderPhone{+31 123 456789}
%\setSenderWeb{odc.gewis.nl}


%% Set the recipient (required parameters have to be set, but can be set as empty string if needed)
\setType{Samenwerkingsvoorstel}           % (optional) Set the Type which is printed above the address - e.g. "Confidential"
\setRecipient{%{company}
} % (required) Set the recipient for this letter
\setAttn{%{contactperson}
}             % (optional) Set the "Attn."/"T.a.v." line in the address
\setStreet{%{street}
}        % (required) Set the address of the recipient (for Dutch addresses usually street + number)
\setPostcode{%{postalcode}
}           % (required) Set postal code
\setCity{%{city}
}           % (required) Set city of recipient
%\setPostcodecity{2020}         % (optional) For countries that don't use postal code + city addressing, you may want to set the line differently
\setCountry{%{country}
}   % (optional) Set the country of the recipient

%% Set letter properties
\setYourreference{%{yourreference}
}        % (optional) Set the reference ("Uw kenmerk") the recipient uses/used
\setMyreference{%{ourreference}
}          % (optional) Set your own reference ("Ons kenmerk")
\setSubject{%{subject}
} % (highly recommended) Set the subject of the letter

\setDate{\today}       % (highly recommended) Set the date the letter is to be sent/was sent

%% Usually you dont wan't the names of persons to be split to multiple lines, so you can define them here
%% You can also define new commands to be able to reuse the names or other variables. Using tildes instead of space, allows you to prevent linebreaks within the command
\newcommand{\GEWIScontact}{%{sender}
\\%{senderfunctie}
}
\hyphenation{%{sender}
}
\newcommand{\COMPANYcontact}{%{contactperson}
}
%% Now it is time to start the letter
\begin{document}
\GEWISfirstpage                 % We want to print the address information of GEWIS on the first page
\printadresenkenmerk            % We want to print the information we just set on the first page

%% We now can write the letter - Note that we can use the variables we set for the names
Geachte %{ontvanger}
,\\[2\baselineskip]
Bij dezen stuur ik u een voorstel tot samenwerking met onze vereniging. Als u nog op- of aanmerkingen heeft, verneem ik deze graag via het emailadres \href{mailto:%{senderemail}
}{%{senderemail}
}.\\[\baselineskip]
Zou u zo vriendelijk willen zijn om me te laten weten of u met dit voorstel akkoord gaat?\\[2\baselineskip]

Met vriendelijke groet,\\[2\baselineskip]
\GEWIScontact 

\section{Samenwerkingsoverzicht}
Het opgemaakte voorstel bestaat uit de volgende items:
 \begin{itemize}
 	%{producten}
 \end{itemize}
\subsection{Financi\"en}
\begin{tabularx}{\textwidth}{X r}\toprule
	Beschrijving           & Bedrag(EUR)\\\midrule
	%{tabelproducten}
    \cmidrule{2-2} \textbf{Totaal} & {\bfseries %{totaalprijs}
}\\\bottomrule
\end{tabularx}
%% Perhaps we want the attachment to start with page 1 again
%% Maybe we want to hide the GEWIS logo in the top right corner (can be undone by \GEWISlargeheading)
%% Note that this is not suitable for printed letters, since we only have preprinted full pages
% \GEWISsmallheading
% \newpage
% \subsection{Financi\"en}
% \begin{tabularx}{\textwidth}{X r}\toprule
% 	Beschrijving           & Bedrag(EUR)\\\midrule
% 	37 Rode kapjes         & 500,00\\
% 	37 Stenen voor de wolf & 800,00\\
% 	\cmidrule{2-2} \textbf{Totaal} & {\bfseries 1.300,00}\\\bottomrule
% \end{tabularx}



%% Maybe we want some other attachment to start with a fresh page count, or we are generating letters in some kind of fancy for-loop
%% We reset the page count
%%We update the recipient and print all the information again





\end{document}
